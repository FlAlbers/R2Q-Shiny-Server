% Options for packages loaded elsewhere
\PassOptionsToPackage{unicode}{hyperref}
\PassOptionsToPackage{hyphens}{url}
%
\documentclass[
]{article}
\usepackage{lmodern}
\usepackage{amssymb,amsmath}
\usepackage{ifxetex,ifluatex}
\ifnum 0\ifxetex 1\fi\ifluatex 1\fi=0 % if pdftex
  \usepackage[T1]{fontenc}
  \usepackage[utf8]{inputenc}
  \usepackage{textcomp} % provide euro and other symbols
\else % if luatex or xetex
  \usepackage{unicode-math}
  \defaultfontfeatures{Scale=MatchLowercase}
  \defaultfontfeatures[\rmfamily]{Ligatures=TeX,Scale=1}
\fi
% Use upquote if available, for straight quotes in verbatim environments
\IfFileExists{upquote.sty}{\usepackage{upquote}}{}
\IfFileExists{microtype.sty}{% use microtype if available
  \usepackage[]{microtype}
  \UseMicrotypeSet[protrusion]{basicmath} % disable protrusion for tt fonts
}{}
\makeatletter
\@ifundefined{KOMAClassName}{% if non-KOMA class
  \IfFileExists{parskip.sty}{%
    \usepackage{parskip}
  }{% else
    \setlength{\parindent}{0pt}
    \setlength{\parskip}{6pt plus 2pt minus 1pt}}
}{% if KOMA class
  \KOMAoptions{parskip=half}}
\makeatother
\usepackage{xcolor}
\IfFileExists{xurl.sty}{\usepackage{xurl}}{} % add URL line breaks if available
\IfFileExists{bookmark.sty}{\usepackage{bookmark}}{\usepackage{hyperref}}
\hypersetup{
  pdftitle={R2Q Maßnahmenkatalog Test},
  hidelinks,
  pdfcreator={LaTeX via pandoc}}
\urlstyle{same} % disable monospaced font for URLs
\usepackage[margin=1in]{geometry}
\usepackage{graphicx,grffile}
\makeatletter
\def\maxwidth{\ifdim\Gin@nat@width>\linewidth\linewidth\else\Gin@nat@width\fi}
\def\maxheight{\ifdim\Gin@nat@height>\textheight\textheight\else\Gin@nat@height\fi}
\makeatother
% Scale images if necessary, so that they will not overflow the page
% margins by default, and it is still possible to overwrite the defaults
% using explicit options in \includegraphics[width, height, ...]{}
\setkeys{Gin}{width=\maxwidth,height=\maxheight,keepaspectratio}
% Set default figure placement to htbp
\makeatletter
\def\fps@figure{htbp}
\makeatother
\setlength{\emergencystretch}{3em} % prevent overfull lines
\providecommand{\tightlist}{%
  \setlength{\itemsep}{0pt}\setlength{\parskip}{0pt}}
\setcounter{secnumdepth}{-\maxdimen} % remove section numbering
\usepackage{tabularx}
\usepackage{fontspec}
\setmainfont{Calibri Light}
\usepackage{array}

\title{R2Q Maßnahmenkatalog Test}
\author{}
\date{\vspace{-2.5em}}

\begin{document}
\maketitle

\hypertarget{xcvbjkjhgfdsdfghj}{%
\section{xcvbjkjhgfdsdfghj}\label{xcvbjkjhgfdsdfghj}}

\hypertarget{kurzinformation}{%
\subsection{\texorpdfstring{\emph{Kurzinformation}}{Kurzinformation}}\label{kurzinformation}}

\hypertarget{kurzbeschreibung}{%
\subsubsection{\texorpdfstring{\emph{Kurzbeschreibung}}{Kurzbeschreibung}}\label{kurzbeschreibung}}

R\_001\_Gründach

\hypertarget{umsetzungsbeispiel}{%
\subsubsection{\texorpdfstring{\emph{Umsetzungsbeispiel}}{Umsetzungsbeispiel}}\label{umsetzungsbeispiel}}

\begin{center}
\begin{tabularx}{\textwidth}{|X|X|X|X|X|}
\hline
\multicolumn{5}{|c|}{\textbf{R\_001\_Gründach}} \tabularnewline
\hline
[ ] Regenwasser & [ ] Schmutzwasser & [ ] Baustoffe & [ ] Energie & [ ] Fläche \\
\hline
\end{tabularx}
\end{center}

\begin{center}
\begin{tabular}{|l|l|l|l|l|}
\hline
\multicolumn{5}{|c|}{\textbf{Wirkung/Funktion}} \tabularnewline
\hline
Wirkung/Funktion & Schmutzwasser & Baustoffe & Energie & Fläche \\
\hline
[ ] Klimaregulation & [ ] Regenwasser & [ ] Regenwasser & [ ] Regenwasser & [ ] Regenwasser \\
\hline
[ ] Wasserregulation & [ ] Schmutzwasser & [ ] Schmutzwasser & [ ] Schmutzwasser & [ ] Schmutzwasser \\
\hline
[ ] Wasserreinigung & [ ] Baustoffe & [ ] Baustoffe & [ ] Baustoffe & [ ] Baustoffe \\
\hline
[ ] Erhaltung biol. Vielfalt & [ ] Energie & [ ] Energie & [ ] Energie & [ ] Energie \\
\hline
\end{tabular}
\end{center}

\begin{center}
\begin{tabularx}{\textwidth}{|X|X|X|}
\hline
\multicolumn{3}{|c|}{\textbf{Wirkung/Funktion}} \tabularnewline
\hline
[ ] Gebäudeebene & [ ] Grundstückebene & [ ] Quartiersebene \\
\hline
\multicolumn{3}{|c|}{\textbf{Hinweis:} Nachwachsender Rohstoff} \\
\hline
\end{tabularx}
\end{center}

\begin{center}
\begin{tabular}{|c|c|c|}
\hline
\multicolumn{3}{|c|}{\textbf{Wirkung/Funktion}} \tabularnewline
\hline
[ ] geringer Bedarf & [ ] mittlerer Bedarf & [ ] hoher Bedarf \\
\hline
\multicolumn{3}{|c|}{Infrastrukturversorgung   Wirkung/Funktion} \\
\hline
\end{tabular}
\end{center}

\begin{center}
\begin{tabular}{|c|c|c|c|c|c|}
\hline
\multicolumn{6}{|c|}{\textbf{Wirkung/Funktion}} \tabularnewline
\hline
Wirkung/Funktion & Nutzungsvielfalt  &Wirkung/Funktion & Einsparung natürlicher Ressourcen  & Wirkung/Funktion & Luftreinhaltung \\
\hline
\multicolumn{6}{|c|}{\textbf{Hinweis:} Biodiversität} \\
\hline
\end{tabular}
\end{center}

\begin{center}
\begin{tabular}{|c|c|c|}
\hline
\multicolumn{3}{|c|}{\textbf{Wirkung/Funktion}} \\
\hline
[ ]Wirkung/Funktion & [ ] Wirkung/Funktion  & [ ] Wirkung/Funktion \\
\hline
\multicolumn{3}{|c|}{\textbf{Hinweis:} Trinwassereinsparung} \\
\hline
\end{tabular}
\end{center}

\hypertarget{detailinformationen}{%
\subsection{\texorpdfstring{\emph{Detailinformationen}}{Detailinformationen}}\label{detailinformationen}}

\hypertarget{funktionsbeschreibung-aufbau}{%
\subsubsection{\texorpdfstring{\emph{Funktionsbeschreibung \&
Aufbau}}{Funktionsbeschreibung \& Aufbau}}\label{funktionsbeschreibung-aufbau}}

Schmutzwasser

\hypertarget{systemskizze}{%
\subsubsection{\texorpdfstring{\emph{Systemskizze}}{Systemskizze}}\label{systemskizze}}

\hypertarget{planung-bemessung-und-rechtliche-aspekte}{%
\subsubsection{Planung, Bemessung und rechtliche
Aspekte}\label{planung-bemessung-und-rechtliche-aspekte}}

Wärme

\hypertarget{aufwand}{%
\subsubsection{Aufwand}\label{aufwand}}

Brennstoffe

\hypertarget{weitergehende-hinweise}{%
\subsubsection{Weitergehende Hinweise}\label{weitergehende-hinweise}}

Erzeugung

\hypertarget{ressourcenuxfcbergreifende-aspekte}{%
\subsubsection{Ressourcenübergreifende
Aspekte}\label{ressourcenuxfcbergreifende-aspekte}}

\begin{center}
\begin{tabularx}{\textwidth}{|X|X|}
\hline
\multicolumn{1}{|c|}{\textbf{Ressource}} & \multicolumn{1}{|c|}{\textbf{Angaben}} \tabularnewline
\hline
\multicolumn{2}{|c|}{ \textbf{SYNERGIEN} } \tabularnewline
\hline
Regenwasser & Verteilung\\
\hline
Schmutzwasser & Verbrauch\\
\hline
Baustoffe & NA\\
\hline
Energie & NA\\
\hline
Fläche & NA\\
\hline
Ökobilanz & NA\\
\hline
\multicolumn{2}{|c|}{ \textbf{ZIELKONFLIKTE} } \\
\hline
Regenwasser & NA\\
\hline
Schmutzwasser & NA\\
\hline
Baustoffe & NA\\
\hline
Energie & NA\\
\hline
Fläche & NA\\
\hline
Ökobilanz & NA\\
\hline
\end{tabularx}
\end{center}

\begin{center}
\begin{tabularx}{\textwidth}{|X|X|}
\hline
\multicolumn{1}{|c|}{\textbf{Vorteile}} & \multicolumn{1}{|c|}{\textbf{Nachteile}} \tabularnewline
\hline
NA & NA\\
\hline
NA & NA\\
\hline
NA & NA\\
\hline
NA & NA\\
\hline
NA & NA\\
\hline
\end{tabularx}
\end{center}

\hypertarget{umsetzungsbeispiele}{%
\subsubsection{Umsetzungsbeispiele}\label{umsetzungsbeispiele}}

NA

\hypertarget{literatur}{%
\subsubsection{Literatur}\label{literatur}}

NA

\end{document}
